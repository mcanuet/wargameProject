\subsection{Tournois}
La fonction tournois prend en paramètre une population d'armées et possède 7 arguments par défaut ( quelle IA, quel module utiliser pour les règles, quel plateau utiliser, combien de fois executer chaque combat, sauvegarde ou non de données ainsi que le fichier où sauvegarder ces données ). \\
Le tournois permet de faire s'affronter plusieurs armées et de les classer par ordre du nombre de victoires. Il est principalement composé de deux boucles for : une première pour parcourir la liste des armées donnée en paramètre, la deuxième pour effectuer plusieurs fois un combat. Au début de chaque combat, si on a donné en paramètre à la fonction tournois save=True, la composition des deux armées, ainsi que la seed du combat, est écrite dans une variable dont le contenu sera sauvegardé dans un fichier précisé en paramètre de la fonction. En fin de combat la composition finale des armées est aussi sauvegardée à la même fin.\\
Une fois que le tournois entre toutes les armées est terminé on écrit dans le fichier de sauvegarde le nombre total de victoire de chaque armée. \\
Une fois toutes les informations enregistrées dans le fichier les populations sot rangées par ordre de nombre de victoires et la fonction renvoie la liste des armées classée ainsi que les informations qui ont été sauvegardées si backData=True.\\

\subsection{Algorithme génétique}
L'algorithme génétique est un objet qui permet à d'effectuer un certain nombre de combats et de sélectionner les armées les plus optimisées pour le jeu. Pour cela on donne à l'algorithme le nombre de tours que l'algorithme devra effectuer, le nombre d'armées qui seront utilisées au départ de l'algorithme, le pourcentage d'armées qui resteront inchangées, le pourcentage d'armées qui seront mutées, le pourcentage d'armées sui seront croisées, et un paramètre qui indique si des armées 'pures' seront utilisées durant le premier tour de l'algorithme. On entend par armées 'pures' des armées constituées d'un seul type d'unités. \\
L'algorithme commence donc par générer le nombre requis d'armées : suivant les paramètres donnés soit les armées sont générées de manière totalement aléatoires, soit on y ajoute les armées 'pures'. \\
Ensuite l'algorithme utilise la fonction tournois pour faire s'affronter les armées et determiner lesquels sont les meilleures. Une fois le classement effectué l'algorithme met dans une nouvelle liste d'armées les meilleurs armées du tournois qui restent inchangées. L'algortihme procède ensuite à des mutations sur certaines armées et produit de nouvelles armées en effectuant des croisements entre d'autres armées. Une fois tout ceci effectuer l'algorithme boucle jusqu'à ce qu'il ait effectuer le nombre d'itération requis. Après cela l'algorithme sauvegarde les informations concernant les mutations et  les croisements qu'il a effectués et affiche le temps qu'il a mis à s'executer.